\section{Contexto de pesquisa}

%Exemplo de contexto que a Joelma deu na sala de aula. Usar como exemplo
%\par Este sistema volta-se para implantação e execução em todas as empresas de pequeno porte do ramo varejista no sul de minas gerais que tenha apresentado a necessidade de um controle mais rigoroso da sua entrada e saída de mercadorias

%Segundo a Joelma e o Márcio este seria o contexto da pesquisa: As pessoas que buscam mão de obra para determinados tipos de trabalho (contratantes) e as pessoas que disponibilizam tais mãos de obra (contratados ou prestadores de serviços).

%Antigo Contexto que a Joelma disse que estava muito geral na correção do pré-projeto
%\par Esta pesquisa terá como foco todas as pessoas que necessitam de mão de obra temporária para realizar tarefas domésticas e rotineiras, além daquelas que não ocorrem com tanta intensidade. Uma vez que o sistema será desenvolvido em uma plataforma \textit{web}, todas as pessoas terão fácil acesso ao serviço, sendo necessário apenas possuir uma comunicação com a internet.

\par Esta pesquisa terá como foco todas as pessoas da região do sul de Minas Gerais, que necessitam de determinados tipos de mão de obra (contratantes), bem como para aqueles que necessitam de um espaço para divulgar esta oferta (contratados ou prestadores de serviços). Tomando por base que encontrar este tipo de profissional tem se tornado uma tarefa cada vez mais complicada, que acaba gerando transtornos na vida da população, pois, em muitos casos, a falta de opções leva a contratações equivocadas e infelizes.

\par Como o sistema será desenvolvido em uma plataforma \textit{web}, todas as pessoas neste contexto terão fácil acesso ao serviço, permitindo a disseminação desta ferramenta, sendo necessário apenas possuir uma comunicação com a internet.
