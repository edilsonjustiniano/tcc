
\chapter{DISCUSSÃO DOS RESULTADOS} 

\par O desenvolvimento deste trabalho teve como objetivo desenvolver um sistema que pudesse auxiliar em contratações de mão de obra temporária, fazendo uso do banco do dados orientado a grafos. Por meio do desenvolvimento do \textit{software}, utilizando as linguagens de programação HTML, CSS, Javascript e Java, juntamente com o \textit{framework} Angular JS, foi possível desenvolver este sistema, iniciando primeiramente com o objetivo de analisar os conceitos e as funcionalidades do banco de dados orientado a grafos.

\par Web service (comunicação com o banco de dados)
\par Telas de navegação (mudanças de tecnologia)
\par Banco de dados (comentar sobre a mudança na maneira de utilização)
\par Linguagem Java (falta de documentação em PHP);


\par Telas de navegação

\par Segundo Shackel (1991), a usabilidade de um sistema pode ser definida como a possibilidade do sistema ser utilizado facilmente e eficazmente pela gama de usuários aos quais ele se destina na realização de determinadas tarefas, dentro de contextos específicos.  O design de uma tela é parte integrante de um produto online e é de extrema importância no planejamento visual. Por meio dele é possível criar uma experiencia amigável ao usuário. Para auxiliar no desenvolvimento das telas, foram escolhidas tecnologias  

