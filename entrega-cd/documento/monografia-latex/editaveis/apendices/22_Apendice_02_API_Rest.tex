\chapter*{Apêndice 2: API Rest}
\label{apendice:api_rest}

Nesse Apêndice será apresentado o contrato de serviços providos pelo \textit{Web Service} desenvolvido neste trabalho. Os serviços foram divididos em categorias, a fim de facilitar a compreensão dos contratos.

\section*{Estados e capitais}

\begin{lstlisting} [style=custom_XML,title={Contrato de serviço referente a estados e capitais. \textbf{Fonte:} Elaborado pelos autores.}, label=list:contrato_estados] 	
<application xmlns="http://wadl.dev.java.net/2009/02">
	<doc xml:lang="en" title="http://localhost:8080"/>
	<resources base="http://localhost:8080">
		<resource path="WebService/uf" id="uf">
			<doc xml:lang="en" title="uf"/>
			<method name="GET" id="uf">
				<doc xml:lang="en" title="uf"/>
				<request/>
				<response>
					<representation mediaType="application/json"/>
				</response>
			</method>
		</resource>
	<resources>
</application>
\end{lstlisting}


\section*{Cidades}

\begin{lstlisting} [style=custom_XML,title={Contrato de serviço referente a cidades. \textbf{Fonte:} Elaborado pelos autores.}, label=list:contrato_cidade] 	
<application xmlns="http://wadl.dev.java.net/2009/02">
	<doc xml:lang="en" title="http://localhost:8080"/>
	<resources base="http://localhost:8080">
		<resource path="WebService/city/cities/{state}" id="city">
			<doc xml:lang="en" title="city"/>
			<param name="state" type="xs:string" required="true" 
				default="" style="template" 
				xmlns:xs="http://www.w3.org/2001/XMLSchema"/>
			<method name="GET" id="city-cities">
				<doc xml:lang="en" title="city-cities"/>
				<request/>
				<response>
					<representation mediaType="application/json"/>
				</response>
			</method>
		</resource>
	<resources>
</application>
\end{lstlisting}


\section*{Sessão}

\begin{lstlisting} [style=custom_XML,title={Contrato de serviço referente a sessão do usuário autenticado. \textbf{Fonte:} Elaborado pelos autores.}, label=list:contrato_sessao] 	
<application xmlns="http://wadl.dev.java.net/2009/02">
	<doc xml:lang="en" title="http://localhost:8080"/>
	<resources base="http://localhost:8080">
		<resource path="WebService/session/userinfo/{token}" id="session">
			<doc xml:lang="en" title="session"/>
			<param name="token" type="xs:string" required="false" 
			default="" style="template" 
			xmlns:xs="http://www.w3.org/2001/XMLSchema"/>
			<method name="POST" id="session-login">
				<doc xml:lang="en" title="session-login"/>
				<request>
					<representation mediaType="application/json"/>
				</request>
				<response>
					<representation mediaType="application/json"/>
				</response>
			</method>
			<method name="GET" id="session-userinfo">
				<doc xml:lang="en" title="session-userinfo"/>
				<request/>
				<response>
					<representation mediaType="application/json"/>
				</response>
			</method>
		</resource>
	</resources>
</application>
\end{lstlisting}

\section*{Serviços}

\begin{lstlisting} [style=custom_XML,title={Contrato de serviço referente a serviços. \textbf{Fonte:} Elaborado pelos autores.}, label=list:contrato_servicos] 	
<application xmlns="http://wadl.dev.java.net/2009/02">
	<doc xml:lang="en" title="http://localhost:8080"/>
	<resources base="http://localhost:8080">
		<resource path="WebService/services/service/{name}" id="services">
			<doc xml:lang="en" title="services"/>
			<param name="name" type="xs:string" required="true"
			 default="" style="template" 
			 xmlns:xs="http://www.w3.org/2001/XMLSchema"/>
			<method name="GET" id="services-service">
				<doc xml:lang="en" title="services-service"/>
				<request/>
				<response>
					<representation mediaType="application/json"/>
				</response>
			</method>
		</resource>
	</resources>
</application>
\end{lstlisting}

\section*{Provedores de serviços}

\begin{lstlisting} [style=custom_XML,title={Contrato de serviço referente a provedores de serviços. \textbf{Fonte:} Elaborado pelos autores.}, label=list:contrato_provedores_servicos] 	
<application xmlns="http://wadl.dev.java.net/2009/02">
	<doc xml:lang="en" title="http://localhost:8080"/>
	<resource path="WebService/serviceprovider/removeservice" id="serviceprovider">
		<doc xml:lang="en" title="serviceprovider"/>
		<method name="GET" id="serviceprovider-byservice">
			<doc xml:lang="en" title="serviceprovider-byservice"/>
			<request/>
			<response>
				<representation mediaType="application/json"/>
			</response>
		</method>
		<method name="GET" id="serviceprovider-ratingInMyNetworkPartners">
			<doc xml:lang="en" title="serviceprovider-ratingInMyNetworkPartners"/>
			<request/>
			<response>
				<representation mediaType="application/json"/>
			</response>
		</method>
		<method name="GET" id="serviceprovider-ratingInMyCompany">
			<doc xml:lang="en" title="serviceprovider-ratingInMyCompany"/>
			<request/>
			<response>
				<representation mediaType="application/json"/>
			</response>
		</method>
		<method name="GET" id="serviceprovider-ratingInMyCity">
			<doc xml:lang="en" title="serviceprovider-ratingInMyCity"/>
			<request/>
			<response>
				<representation mediaType="application/json"/>
			</response>
		</method>
		<method name="GET" id="serviceprovider-data">
			<doc xml:lang="en" title="serviceprovider-data"/>
			<request/>
			<response>
				<representation mediaType="application/json"/>
			</response>
		</method>
		<method name="GET" id="serviceprovider-myservices">
			<doc xml:lang="en" title="serviceprovider-myservices"/>
			<request/>
			<response>
				<representation mediaType="application/json"/>
			</response>
		</method>
		<method name="POST" id="serviceprovider-addservice">
			<doc xml:lang="en" title="serviceprovider-addservice"/>
			<request>
				<representation mediaType="application/json"/>
			</request>
			<response>
				<representation mediaType="application/json"/>
			</response>
		</method>
		<method name="POST" id="serviceprovider-removeservice">
			<doc xml:lang="en" title="serviceprovider-removeservice"/>
			<request>
				<representation mediaType="application/json"/>
			</request>
			<response>
				<representation mediaType="application/json"/>
			</response>
		</method>
	</resource>
</application>
\end{lstlisting}

\section*{Avaliações de serviços}

\begin{lstlisting} [style=custom_XML,title={Contrato de serviço referente a avaliações de serviços. \textbf{Fonte:} Elaborado pelos autores.}, label=list:contrato_avaliacao_servicos] 	
<application xmlns="http://wadl.dev.java.net/2009/02">
	<doc xml:lang="en" title="http://localhost:8080"/>
	<resource path="WebService/rating/mylastestratings/{token}" id="rating">
		<doc xml:lang="en" title="rating"/>
		<param name="token" type="xs:string" required="true" 
		default="" style="template" 
		xmlns:xs="http://www.w3.org/2001/XMLSchema"/>
		<method name="POST" id="rating-save">
			<doc xml:lang="en" title="rating-save"/>
			<request>
				<representation mediaType="application/json"/>
			</request>
			<response>
				<representation mediaType="application/json"/>
			</response>
		</method>
		<method name="GET" id="rating-mylastestratings">
			<doc xml:lang="en" title="rating-mylastestratings"/>
			<request/>
			<response>
				<representation mediaType="application/json"/>
			</response>
		</method>
	</resource>	
</application>
\end{lstlisting}

\section*{Informações pessoais}

\begin{lstlisting} [style=custom_XML,title={Contrato de serviço referente a informações pessoais. \textbf{Fonte:} Elaborado pelos autores.}, label=list:contrato_informacoes_pessoais] 	
<application xmlns="http://wadl.dev.java.net/2009/02">
	<resource path="WebService/person/persondata/{partner}" id="person">
		<doc xml:lang="en" title="person"/>
		<param name="partner" type="xs:string" required="true"
		 default="" style="template" 
		 xmlns:xs="http://www.w3.org/2001/XMLSchema"/>
		<param name="token" type="xs:string" required="true"
		 default="" style="query" 
		 xmlns:xs="http://www.w3.org/2001/XMLSchema"/>
		<method name="POST" id="person-createaccount-personaldata">
			<doc xml:lang="en" title="person-createaccount-personaldata"/>
			<request>
				<representation mediaType="application/json"/>
			</request>
			<response>
				<representation mediaType="application/json"/>
			</response>
		</method>
		<method name="POST" id="person-createaccount-workdata">
			<doc xml:lang="en" title="person-createaccount-workdata"/>
			<request>
				<representation mediaType="application/json"/>
			</request>
			<response>
				<representation mediaType="application/json"/>
			</response>
		</method>
		<method name="GET" id="person-persondata">
			<doc xml:lang="en" title="person-persondata"/>
			<request/>
			<response>
				<representation mediaType="application/json"/>
			</response>
		</method>
	</resource>
</application>
\end{lstlisting}


\section*{Parceiros}

\begin{lstlisting} [style=custom_XML,title={Contrato de serviço referente a informações de paceiros. \textbf{Fonte:} Elaborado pelos autores.}, label=list:contrato_informacoes_de_parceiros] 	
<application xmlns="http://wadl.dev.java.net/2009/02">
	<resource path="WebService/partner/commonspartner/{partner}" id="partner">
		<doc xml:lang="en" title="partner"/>
		<param name="partner" type="xs:string" required="true" 
		default="" style="template" 
		xmlns:xs="http://www.w3.org/2001/XMLSchema"/>
		<param name="token" type="xs:string" required="true" 
		default="" style="query" 
		xmlns:xs="http://www.w3.org/2001/XMLSchema"/>
		<method name="GET" id="partner-possiblepartners">
			<doc xml:lang="en" title="partner-possiblepartners"/>
			<request/>
			<response>
				<representation mediaType="application/json"/>
			</response>
		</method>
		<method name="GET" id="partner-allpartners">
			<doc xml:lang="en" title="partner-allpartners"/>
			<request/>
			<response>
				<representation mediaType="application/json"/>
			</response>
		</method>
		<method name="POST" id="partner-add">
			<doc xml:lang="en" title="partner-add"/>
			<request>
				<representation mediaType="application/json"/>
			</request>
			<response>
				<representation mediaType="application/json"/>
			</response>
		</method>
		<method name="POST" id="partner-cancel">
			<doc xml:lang="en" title="partner-cancel"/>
			<request>
				<representation mediaType="application/json"/>
			</request>
			<response>
				<representation mediaType="application/json"/>
			</response>
		</method>
		<method name="GET" id="partner-allpartnerrequest">
			<doc xml:lang="en" title="partner-allpartnerrequest"/>
			<request/>
			<response>
				<representation mediaType="application/json"/>
			</response>
		</method>
		<method name="POST" id="partner-acceptpartnerrequest">
			<doc xml:lang="en" title="partner-acceptpartnerrequest"/>
			<request>
				<representation mediaType="application/json"/>
			</request>
			<response>
				<representation mediaType="application/json"/>
			</response>
		</method>
		<method name="POST" id="partner-rejectpartnerrequest">
			<doc xml:lang="en" title="partner-rejectpartnerrequest"/>
			<request>
				<representation mediaType="application/json"/>
			</request>
			<response>
				<representation mediaType="application/json"/>
			</response>
		</method>
		<method name="POST" id="partner-searchnewpartners">
			<doc xml:lang="en" title="partner-searchnewpartners"/>
			<request>
				<representation mediaType="application/json"/>
			</request>
			<response>
				<representation mediaType="application/json"/>
			</response>
		</method>
		<method name="POST" id="partner-searchnewpartnersonlybyname">
			<doc xml:lang="en" title="partner-searchnewpartnersonlybyname"/>
			<request>
				<representation mediaType="application/json"/>
			</request>
			<response>
				<representation mediaType="application/json"/>
			</response>
		</method>
		<method name="GET" id="partner-ismypartner">
			<doc xml:lang="en" title="partner-ismypartner"/>
			<request/>
			<response>
				<representation mediaType="application/json"/>
			</response>
		</method>
		<method name="GET" id="partner-commonspartner">
			<doc xml:lang="en" title="partner-commonspartner"/>
			<request/>
			<response>
				<representation mediaType="application/json"/>
			</response>
		</method>
	</resource>
</application>
\end{lstlisting}


\section*{Últimas atualizações}

\begin{lstlisting} [style=custom_XML,title={Contrato de serviço referente as últimas atualizações. \textbf{Fonte:} Elaborado pelos autores.}, label=list:contrato_ultimas_atualizacoes] 	
<application xmlns="http://wadl.dev.java.net/2009/02">
	<resource path="WebService/feed/lastestratings/{token}" id="feed">
		<doc xml:lang="en" title="feed"/>
		<param name="token" type="xs:string" required="false" 
		default="" style="template" 
		xmlns:xs="http://www.w3.org/2001/XMLSchema"/>
		<method name="GET" id="feed-lastestpartnership">
			<doc xml:lang="en" title="feed-lastestpartnership"/>
			<request/>
			<response>
				<representation mediaType="application/json"/>
			</response>
		</method>
		<method name="GET" id="feed-lastestratings">
			<doc xml:lang="en" title="feed-lastestratings"/>
			<request/>
			<response>
				<representation mediaType="application/json"/>
			</response>
		</method>
	</resource>
</application>
\end{lstlisting}

\section*{Gráficos}

\begin{lstlisting} [style=custom_XML,title={Contrato de serviço referente aos gráficos. \textbf{Fonte:} Elaborado pelos autores.}, label=list:contrato_graficos] 	
<application xmlns="http://wadl.dev.java.net/2009/02">
	<resource path="WebService/report/lastEvaluateInMyCity" id="report">
		<doc xml:lang="en" title="report"/>
		<param name="token" type="xs:string" required="true"
		 style="query" xmlns:xs="http://www.w3.org/2001/XMLSchema"/>
		<param name="service" type="xs:string" required="true"
		 style="query" xmlns:xs="http://www.w3.org/2001/XMLSchema"/>
		<param name="limit" type="xs:string" required="true"
		 style="query" xmlns:xs="http://www.w3.org/2001/XMLSchema"/>
		<method name="GET" id="report-lastEvaluateOfServiceProvider">
			<doc xml:lang="en" title="report-lastEvaluateOfServiceProvider"/>
			<request/>
			<response status="200">
				<representation mediaType="application/json"/>
			</response>
		</method>
		<method name="GET" id="report-lastEvaluateOfServiceInNetwork">
			<doc xml:lang="en" title="report-lastEvaluateOfServiceInNetwork"/>
			<request/>
			<response status="200">
				<representation mediaType="application/json"/>
			</response>
		</method>
		<method name="GET" id="report-lastEvaluate">
			<doc xml:lang="en" title="report-lastEvaluate"/>
			<request/>
			<response status="200">
				<representation mediaType="application/json"/>
			</response>
		</method>
		<method name="GET" id="report-lastEvaluateInMyCity">
			<doc xml:lang="en" title="report-lastEvaluateInMyCity"/>
			<request/>
			<response status="200">
				<representation mediaType="application/json"/>
			</response>
		</method>
	</resource>
</application>
\end{lstlisting}

Com este contrato definido foi possível dar continuidade no processo de desenvolvimento da aplicação.