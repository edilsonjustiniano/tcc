\section{Procedimentos}

\par Para que esta pesquisa seja levada a cabo, torna-se necessário a implementação de algumas ações, as quais são descritas abaixo.

\begin{itemize}
	\item Realizar o levantamento de tecnologias a serem utilizadas;
	\item Fazer um levantamento de teorias em livros da área;
	\item Fazer um levantamento de requisitos;
	\item Definir todas as ações que o usuário poderá realizar, por meio de diagramas como o de caso de uso;
	\item Escrever os diagramas necessários para o início do desenvolvimento do projeto;
	\item Desenvolver a base de dados, de acordo com os requisitos levantados;
	\item Implementar o sistema utilizando as tecnologias descritas no quadro teórico;
	\item Realizar testes, a fim de, evitar que erros passem despercebidos;
	\item Disponibilizar o sistema na \textit{web}.
\end{itemize}

\par Realizando todos os passos descritos acima, será possível obter como resultado final a realização deste projeto.


%Comentar, pois na correção do pré-projeto a Joelma disse que isto seria feito via tópicos e não texto.
%\par Para desenvolver este projeto, usaremos os processos de engenharia de softwares definidos no ICONIX, que consistem em quatro etapas subdivididas em algumas tarefas específicas.

%\par Na primeira etapa, será realizada a análise dos requisitos que serão necessários para o início do projeto. Nesta fase será desenvolvido o modelo de domínio e, posteriormente, os casos de uso.

%\par Após o término da primeira etapa, será feita a análise preliminar, realizando para cada caso de uso, um diagrama de robustez e atualizando paralelamente o modelo de domínio, com os atributos e métodos. Com as informações geradas a partir deste processo, será possível desenvolver a base de dados da aplicação.

%\par Após a segunda etapa concluída será criado, para cada caso de uso, um diagrama de sequencia, contendo detalhes a respeito da futura implementação e novamente, o modelo de domínio gerado na 1ª etapa será atualizado, incluindo a ele os novos métodos que foram coletados neste processo.

%\par Para concluir, o último passo será a implementação, que com base em todos os processos descritos acima, será realizado com segurança, de que este projeto possivelmente irá atender aos requisitos levantados.
