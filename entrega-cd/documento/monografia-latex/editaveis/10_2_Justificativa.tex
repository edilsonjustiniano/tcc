\chapter{JUSTIFICATIVA}

\par Por meio de uma pesquisa informal, realizada com o auxílio de formulários disponibilizados na internet, na região de Pouso Alegre, foi constatado que não há um sistema que possua como principal objetivo localizar determinados tipos de mão de obra nos quais não existam vínculos empregatícios.

\par Este projeto propõem atuar sobre esta limitação, desenvolvendo um software cujo principal objetivo é localizar e apresentar aos usuários, profissionais temporários que possuam credibilidade e boas referências.
	
\par A fim de tornar possível a realização do mesmo serão utilizadas tecnologias gratuitas e de boa aceitação pelo mercado. Desta forma, será possível reduzir o custo de desenvolvimento, possibilitando a sua distribuição aos usuários. Com esta distribuição, seus benefícios terão acesso a uma parcela maior de pessoas, aumentando consideravelmente a facilidade na busca por este tipo de profissional.
	
\par Este projeto também visa agregar valor acadêmico, proporcionando uma base de conhecimentos e explanação de tecnologias atuais e valorizadas, sendo que algumas não fazem parte do escopo do curso, como exemplo, o conceito de banco de dados orientado a grafos e o Neo4J.
	
\par Agregando todos estes benefícios à possibilidade de melhoria contínua e acréscimos de novas funcionalidades, este projeto servirá como base para novos trabalhos acadêmicos.