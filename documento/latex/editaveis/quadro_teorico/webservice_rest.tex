\subsection{\textit{Web service} REST}

Segundo \citeonline{msdn_web_service}, um \textit{Web service} consiste em um programa que disponibiliza serviços e pode ser acessado através do protocolo HTTP, portanto, tais serviços podem ser acessados de diferentes lugares e diferentes dispositivos simultaneamente.

O primeiro\textit{Web service} REST\footnotemark[17] foi criado por Roy Fielding no ano 2000 na universidade da California. Ele foi criado para suceder o tecnologia \textit{Simple Object Access Protocol} - SOAP\footnotemark[18] - a fim de facilitar o uso de tal tecnologia. \cite{ibm_web_service}.

\footnotetext[17]{REST: \textit{Representational State Transfer} - Tecnologia utilizada por \textit{web services}}

\footnotetext[18]{SOAP: \textit{Simple Object Access Protocol} - Tecnologia utilizada por \textit{web services} anteriores aos \textit{Web services} REST.}

Segundo \citeonline{oracle_web_service}, na arquitetura do REST os dados e as funcionalidades são considerados recursos e ambos são acessados por meio de  URIs\footnotemark[19]. Ele funciona baseado no protocolo HTTP, portanto, já possui os mecanismos de segurança, cabeçalhos e respostas embutidos.

\footnotetext[19]{URI: \textit{Uniform Resource Identifier} - \textit{Link} completo para acessar um determinado recurso.}

Para \citeonline{ibm_web_service}, o \textit{Web service} REST segue 4 princípios básicos. São eles: 

\begin{itemize}
	\item Utilização dos métodos HTTP explicitamente;
	\item Orientação à conexão;
	\item Expõe a estrutura de diretório por meio das URIs;
	\item Trabalha com XML, \textit{Javascript Object Notation} - JSON\footnotemark[20] - ou ambos.
\end{itemize}

\footnotetext[20]{JSON: \textit{Javascript Object Notation} - Notação de objetos via Javascript, porém seu uso não é restrito apenas ao Javascript.}

Esta tecnologia foi selecionada para ser utilizada neste trabalho, pois a forma de acesso ao banco de dados Neo4j utiliza uma API REST fornecida pelo próprio banco de dados.