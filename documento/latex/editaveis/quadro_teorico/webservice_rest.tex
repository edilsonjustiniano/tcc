\subsection{\textit{Web Service} REST}

A definição computacional de serviço é um programa que disponibiliza sua funcionalidade através de uma interface denominada contrato de serviço \cite{erl_soa_with_rest}.

\textit{Web Service} de acordo com \citeonline{marzulo_soa_na_pratica}, é uma materialização da ideia de um serviço que é disponibilizado na internet, e que, devido a isto, pode ser acessado em qualquer lugar do planeta e por diferentes tipos de dispositivos. Para ter acesso aos serviços que este \textit{Web Service} disponibiliza o solicitante envia requisições de um tipo anteriormente definido e recebem respostas síncronas ou assíncronas.

\citeonline{marzulo_soa_na_pratica}, afirma que, a implementação de um \textit{Web Service} é relativamente simples, uma vez que, há inúmeras ferramentas que facilitam a implementação do mesmo. Outro fator que permite a um \textit{Web Service} ser mais dinâmico, é possuir uma estrutura interna fracamente acoplada, permitindo assim, mudanças em suas estruturas sem afetar a utilização pelo cliente.

\citeonline{erl_soa_with_rest}, afirma que, o REST\footnotemark[20] é uma das várias implementações utilizadas para criar serviços. Outras implementações também muito conhecidas são: a WSDL\footnotemark[21] e a SOAP\footnotemark[22].

\footnotetext[20]{REST: \textit{Representational State Transfer} - Tecnologia utilizada por \textit{web services}}

\footnotetext[21]{WSDL: \textit{Web Service Description Language} - Padrão de mercado utilizado para descrever \textit{Web Services}.}

\footnotetext[22]{SOAP: \textit{Simple Object Access Protocol} - Tecnologia utilizada por \textit{web services} anteriores aos \textit{Web services} REST.}

O primeiro\textit{Web Service} REST foi criado por Roy Fielding no ano 2000 na universidade da California. Ele foi criado para suceder o tecnologia SOAP, a fim dex facilitar o uso de tal tecnologia. \cite{ibm_web_service}.

Segundo \citeonline{oracle_web_service}, na arquitetura do REST os dados e as funcionalidades são considerados recursos e ambos são acessados por meio de  URIs\footnotemark[23]. Ele funciona baseado no protocolo HTTP, portanto, já possui os mecanismos de segurança, cabeçalhos e respostas embutidos.

\footnotetext[23]{URI: \textit{Uniform Resource Identifier} - \textit{Link} completo para acessar um determinado recurso.}

Para \citeonline{ibm_web_service}, o \textit{Web service} REST segue quatro princípios básicos. São eles: 

\begin{itemize}
	\item Utilização dos métodos HTTP explicitamente;
	\item Orientação à conexão;
	\item Expõe a estrutura de diretório por meio das URIs;
	\item Trabalha com -\textit{Extensible Markup Language} - XML\footnotemark[24], \textit{Javascript Object Notation} - JSON\footnotemark[25] - ou ambos.
\end{itemize}

\footnotetext[24]{XML: \textit{Extensible Markup Language} - Tecnologia utilizada para transferência de dados, configuração, entre outras atividades.}

\footnotetext[25]{JSON: \textit{Javascript Object Notation} - Notação de objetos via Javascript, porém seu uso não é restrito apenas ao Javascript.}

Esta tecnologia foi selecionada para ser utilizada neste trabalho, pois a forma de acesso ao banco de dados Neo4j utiliza uma API REST fornecida pelo próprio Neo4j.