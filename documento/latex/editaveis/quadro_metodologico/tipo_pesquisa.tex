\section{Tipo de pesquisa}

\par Para \citeonline[p. 31]{padua_metodologia_pesquisa}, pesquisa é:

\begin{citacao}
	Toda atividade voltada para a solução de problemas; como atividade de busca, indagação, investigação, inquirição da realidade, e a atividade que visa nos permitir, no âmbito da ciência, elaborar um  conhecimento, ou um conjunto de conhecimentos, que nos auxilie na compreensão desta realidade e nos oriente em nossas ações.
\end{citacao}

% Comentário, pois estava repetindo demais o fato de utilizar  apesquisa aplicada no desenvolvimento do trabalho. 
%\par O tipo de pesquisa aplicada foi escolhido para o desenvolvimento da pesquisa, pois conforme \citeonline[p. 32]{cooper_schindler_metodos_pesquisa_administracao}, ela ``tem uma ênfase prática na solução de problemas, embora a solução de problemas nem sempre seja gerada por uma circunstância negativa.''

\par Conforme \citeonline[p. 32]{cooper_schindler_metodos_pesquisa_administracao}, a pesquisa aplicada: ``tem uma ênfase prática na solução de problemas, embora a solução de problemas nem sempre seja gerada por uma circunstância negativa''.

\par Seguindo esta ideia, este tipo de pesquisa será utilizado no desenvolvimento deste projeto, pois, o objetivo é gerar uma solução para o problema de localização de mão de obra, por meio de um sistema \textit{web}. Este projeto adequa-se perfeitamente ao tipo de pesquisa aplicada, uma vez que o mesmo busca analisar e gerar uma possível solução para o problema em destaque.

%Parágrafo utilizado no pré-projeto. Foi corrigido por Edilson no dia 02/04/15 e criado o parágrafo acima
%\par Este tipo de pesquisa será utilizada no desenvolvimento deste projeto, pois a mesma busca analisar o problema e gerar uma solução para o mesmo através de um aplicativo ou serviço.  Neste caso, será o desenvolvimento de um sistema \textit{web} que auxilia na busca por profissionais temporários.
