\section{Instrumentos}

\par Os instrumentos de pesquisa são as ferramentas usadas para a coleta de dados. Como afirma \citeonline[p. 117]{marconi_lakatos_metodologia_trabalho_cientifico}, os instrumentos de pesquisa abrangem “desde os tópicos de entrevista, passando pelo questionamento e formulário, até os testes ou escala de medida de opiniões e atitudes”. Eles são de suma importância para o desenvolvimento de um projeto, pois visam levantar o máximo de informações possíveis para nortear as tomadas de decisões. Para a realização desta pesquisa foram utilizados questionários e análise documental.

\par Segundo \citeonline{gil_como_elaborar_projeto_de_pesquisa}, o questionário é um dos procedimentos mais utilizados para obter informações, pois é uma técnica de custo razoável, apresenta as mesmas questões para todas as pessoas, garante o anonimato e pode conter as questões para atender a finalidades especificas de uma pesquisa. Se aplicada criteriosamente, esta técnica apresenta elevada confiabilidade. Podem ser desenvolvidos para medir opiniões, comportamento, entre outras questões, também pode ser aplicada individualmente ou em grupos.

\par Para este trabalho, foi desenvolvido um questionário informal, aplicado de forma individual, disponibilizado no ambiente virtual, o qual foi respondido com o intuito de analisar se a pesquisa aqui pretendida seria bem aceita por pessoas de diferentes perfis sociais. %A seguir são discriminados os resultados obtidos.

%Aqui colocar o resultado da análise do questionário

\par A análise documental consiste em identificar, verificar e apreciar os documentos, atendendo a uma finalidade específica, que visa extrair uma informação objetiva da fonte. Para este trabalho foi utilizada a análise documental como material de apoio, que norteou tanto o desenvolvimento teórico quanto o prático.

\par Feita a escolha dos instrumentos, foram definidos os procedimentos necessários para que esta pesquisa fosse realizada. Estes procedimentos serão descritos na próxima seção.


