\subsection{Angular JS}

Segundo \citeonline{green_seshadri_angularjs} o framework Angular JS foi criado para facilitar o desenvolvimento de aplicativos \textit{web}, pois através dele, é possível criar um aplicativo \textit{web} com pucas linhas.

Para o criador do Angular JS, Miško Hevery, o que o motivou a criar o Angular JS, foi a necessidade de ter que reescrever determinados trechos de códigos em todos os outros projetos. O que para ele estava se tornando chato. Portanto, Miško decidiu criar algo para facilitar o desenvolvimento de aplicativos \textit{web} de uma forma que ninguém havia pensado antes.

O Angular JS é um framework \textit{Model-View-Controller} - MVC\footnotemark[27] - escrito em Javascript. Ele é executado pelos navegadores de internet e ajuda os desenvolvedores a escreverem modernos aplicativos \textit{web} \cite{kozlowski_darwin_mastering_web_application_angular_js}.

\footnotetext[27]{MVC: \textit{Model-View-Controller} - \textit{Design pattern}.}

Devido a estas facilidades que o Angular JS nos traz, é que o mesmo foi escolhido para ser utilizado, a fim de, auxiliar no desenvolvimento, não apenas das páginas web e seus respectivos conteúdos, como também na lógica e comunicação com o \textit{Web Service} REST.
