\subsection{HTML 5}

Segundo \citeonline{w3c_html_fundamentals}, \textit{Hypertext Markup Language}  - HTML\footnotemark[23] - é a linguagem usada para descrever o conteúdo das páginas \textit{web}. Ela utiliza marcadores denominados \textit{tags} para identificar aos navegadores de internet como eles devem interpretar tal documento.

\citeonline{silva_css_3}, afirma que, o HTML foi criado única e exclusivamente para ser uma linguagem de marcação e estruturação de documentos (páginas) \textit{web}. Portanto, não cabe a ele definir os aspectos dos componentes como cores, espaços, fontes, etc.

\footnotetext[23]{HTML: \textit{Hypertext Markup Language} - Linguagem usada para descrever o conteúdo das páginas \textit{web}.}

A \citeonline{w3c_html_fundamentals}, afirma que, a primeira versão do HTML foi criada no ano de 1991 pelo inventor da \textit{web}, Tim Berners-Lee. A partir desta versão, o HTML foi e continua em constante atualização e hoje se encontra na sua oitava versão. As oito versões são: HTML, HTML +, HTML 2.0, 3.0, 3.2, 4.0, 4.01 e a versão atual é a 5.

Ao longo dos anos e da evolução propriamente dita do HTML, novas \textit{tags} foram criadas, padrões adotados, e claro, novas versões foram criadas, até que, em maio de 2007  o \textit{World Wide Web Consortium} - W3C\footnotemark[24] - confirma a decisão de voltar a trabalhar na atual versão do HTML, também conhecida por HTML 5 \cite{w3c_html_fundamentals}.

\footnotetext[24]{W3C:  \textit{World Wide Web Consortium} - Consórcio internacional formado por empresas, instituições, pesquisadores, desenvolvedores e público em geral, com a finalidade de elevar a web ao seu 	potencial máximo.}

\citeonline{silva_html5}, afirma que, em novembro daquele mesmo ano, o W3C publicou uma nota contendo uma série de diretrizes que descreve os princípios a serem seguidos ao desenvolver utilizando o HTML 5 em algumas áreas. Tais princípios, permitiram a esta versão, maior segurança, maior compatibilidade entre navegadores e interoperabilidade entre diversos dispositivos.

Por estes motivos, somado ao fato de que, ao utilizar esta tecnologia é possível obter uma maior flexibilidade no desenvolvimento das páginas \textit{web}, Esta tecnologia em conjunto com outras como: CSS 3, Javascript e Angular JS, cujo suas descrições serão apresentadas nas próximas seções deste trabalho foi selecionada para ser utilizada neste projeto.