%\chapter*{Introdução}
\begin{center}
	\vspace{1.2em}
	\textbf{\large INTRODUÇÃO}
	\vspace{2.9em}
\end{center}
\thispagestyle{empty}

\addcontentsline{toc}{chapter}{Introdução}
\stepcounter{chapter} %incrementa o número do capítulo

\par Com a constante evolução tecnológica é possível notar que, a cada dia, mais pessoas estão sendo inseridas em um mundo globalizado. Pertencer a este meio tem mudado completamente a maneira de se realizar tarefas, uma vez que, a revolução tecnológica busca facilitar o que até então era trabalhoso. Atualmente, nos deparamos com situações nas quais as pessoas desempenham mais de um papel, dividindo o seu tempo entre tarefas profissionais, pessoais e aquelas que chamamos de rotineiras as quais muitas vezes, não são realizadas por elas e sim por terceiros. % vem otimizando onde nada para, ocorre um aumento da carga de trabalho, pressões por resultados e mudanças constantes no mercado; é comum ouvir pessoas falando que as horas do dia não são suficientes, deixando a impressão que o tempo acelerou e as pessoas não conseguem mais conciliar suas atividades nas 24 horas do dia. 
%Muitas vezes, essa falta de tempo faz com que a população opte por contratação de um profissional temporário, que auxilie nas tarefas domésticas e rotineiras. Este tipo de profissional é capaz de 
Um profissional terceirizado é capaz de cuidar de todas as tarefas extras que não cabem na rotina, no entanto, encontrá-los tem se tornado cada vez mais difícil, uma vez que confiar a sua residência ou, ainda, a sua intimidade a uma pessoa não conhecida gera muita insegurança. Ainda, vale ressaltar que este tipo de trabalho está cada vez mais escasso e raro no cenário atual.

\par Com o advento da internet, e mais tarde sua popularização, uma série de aplicações para diferentes fins vem sendo desenvolvidas. Estes aplicativos tem como finalidade apresentar soluções para os mais diversos tipos de problemas. É possível notar que as redes sociais geram um grande impacto no modelo de vida que seguimos. Hoje, sabemos que tudo e todos estão conectados, direta ou indiretamente, e que as informações são trocadas a uma velocidade surpreendente. Precisamos obter informações de forma clara e rápida \cite{barbosa_why_people_use_social_network}.

\par% Com intuito de obter acesso à informação de maneira instantânea, a migração dos antigos computadores pessoais (\textit{desktops}) para dispositivos cada vez mais portáteis tornou-se comum. Podemos citar, como exemplo, a criação dos \textit{tablets} que proporcionaram uma grande revolução tecnológica, pois, a partir desta invenção as pessoas passaram a ter liberdade de acesso, uma vez que as informações se encontram na palma de suas mãos. Seguindo esta tendência de mobilidade, os \textit{smartphones} foram desenvolvidos.

\par %O sucesso obtido por estes dispositivos móveis, somado a evolução das redes de dados 3G e 4G, permitiu a conexão entre as pessoas de qualquer lugar e a qualquer momento bastando alguns toques, tornando assim ainda mais fácil localizá-las e comunicar-se com elas.

\par %A pesquisa realizada pelo Instituto Nielsen reafirma o contexto anteriormente mencionado. Ela comprova que o tempo gasto pelos usuários em seus \textit{smartphones} é maior do que em seus computadores pessoais, sendo que boa parte deste tempo é utilizado para acesso às redes sociais. Isto ocorre pois os dispositivos móveis atuais são capazes de substituir os computadores pessoais em uma série de atividades, tornando-os um computador de mão disponível a qualquer instante. A pesquisa, ainda demonstra o sucesso das redes sociais que, segundo a mesma, se deve a disseminação dos dispositivos móveis e ao modelo de vida que grande parte das pessoas segue atualmente.

\par Com a ampliação desta comunicação, somada a evolução das tecnologias, houve a necessidade de se realizar algumas melhorias voltadas aos bancos de dados, a fim de otimizar as tarefas realizadas por ele, levando assim, a criação de uma nova geração de bancos de dados, denominada NoSQL\footnotemark[1].

\footnotetext[1]{NOSQL: \textit{Not Only SQL} - Bancos de dados que utilizam não somente os recursos de Structure Query Language - SQL, a fim de obter melhor performance.}

\par Dentre esta nova geração, o banco de dados orientado a grafos vem se destacando por apresentar grande potencial para manipulação de dados em diferentes áreas como: relacionamento interpessoal, biológicas, rotas geográficas, entre outras.

\par \citeonline{tcc_univas_penha_carvalho_recomendacao_filmes} utilizaram esta tecnologia para desenvolver uma aplicação, cujo principal objetivo é recomendar filmes aos seus usuários.

\par \citeonline{tcc_univas_silva_pires_comparacao_banco_dados} realizaram uma comparação entre banco de dados orientado a grafos e os relacionais a fim de apresentar as vantagens que estes possuem sob os bancos de dados relacionais.

\par \citeonline{junid_tahir_majid_idros_potential_graph_thory_for_dna_sequence_alignment} utilizaram a teoria dos grafos para tentar otimizar o processo de alinhamento da sequência de DNA a fim de determinar a região comum entre duas ou mais sequências deste.

\par Por meio de uma pesquisa informal, realizada com o auxílio de formulários disponibilizados na internet, na região de Pouso Alegre, foi constatado que não há um sistema que possua como principal objetivo localizar determinados tipos de mão de obra nos quais não existam vínculos empregatícios.

\par Este projeto propõem atuar sobre esta limitação, desenvolvendo um software cujo principal objetivo é localizar e apresentar aos usuários, profissionais temporários que possuam credibilidade e boas referências.

\par A fim de tornar possível a realização do mesmo serão utilizadas tecnologias gratuitas e de boa aceitação pelo mercado. Desta forma, será possível reduzir o custo de desenvolvimento, possibilitando a sua distribuição aos usuários. Com esta distribuição, seus benefícios terão acesso a uma parcela maior de pessoas, aumentando consideravelmente a facilidade na busca por este tipo de profissional.


\par Com o intuito de tentar solucionar o problema relacionado a busca por mãos de obra temporárias, este trabalho tem como proposta apresentar uma possível solução, através de uma aplicação \textit{web}, seguindo o modelo de negócio das redes sociais, utilizando o conceito de orientação a grafos, a fim de gerar a familiarização desta ferramenta, uma vez que este tipo de serviço já está intrínseco na atualidade.

\par Este projeto também visa agregar valor acadêmico, proporcionando uma base de conhecimentos e explanação de tecnologias atuais e valorizadas, sendo que algumas não fazem parte do escopo do curso.

\par Agregando todos estes benefícios à possibilidade de melhoria contínua e acréscimos de novas funcionalidades, este projeto servirá como base para novos trabalhos acadêmicos.