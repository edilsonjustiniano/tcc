\section{Procedimentos}

\par Para que esta pesquisa fosse levada a cabo, tornou-se necessário a implementação de algumas ações, as quais serão detalhadas.

\par O início da pesquisa deu-se através da escolha do tema, seguido pelo levantamento das tecnologias que seriam utilizadas. A princípio, foi definido um escopo contendo algumas tecnologias que foram ministradas no ambiente acadêmico, o que diminui a curva de aprendizado, no entanto, foi preciso agregar alguns conhecimentos novos, onde foi desprendido um tempo a mais para estudo e realização de pequenos testes. As tecnologias que foram empregadas no desenvolvimento deste trabalho são as seguintes: a linguagem de programação JAVA, o banco de dados orientado a grafos Neo4j, juntamente com a API Cypher, Tomcat, Primefaces e JSF, sendo que no decorrer do desenvolvimento prático, viu-se a necessidade de substituir as duas ultimas tecnologias citadas pelas linguagens HTML, CSS, Javascript e Angular JS. As tecnologias que acompanharam o desenvolvimento deste trabalho até a sua conclusão estão descritas no quadro teórico desta pesquisa.

\par Para garantir que as tecnologias selecionadas seriam a melhor escolha no desenvolvimento deste trabalho, foram realizados alguns testes por meio de aplicações simples. Os testes foram focados na avaliação do comportamento do banco de dados orientado a grafos aplicado ao contexto desta pesquisa, cujo objetivo foi desenvolver uma aplicação de busca de mão de obra baseada em uma rede de relacionamentos. Estes testes também foram realizados como fins didáticos, a fim de se familiarizar com as tecnologias utilizadas.

\par Para garantir a qualidade desta pesquisa, o ICONIX foi escolhido como a metodologia de desenvolvimento de \textit{software}, desempenhando um papel fundamental na organização do trabalho. Sua abordagem proveu uma sequencia de procedimentos, que foram seguidos conforme o necessário, levando a construção de uma aplicação estável. Como relatado no quadro teórico, foram seguidas as quatro fases definidas pelo ICONIX.
\par Na primeira fase, definida como análise de requisitos, foi realizado o levantamento das informações pertinentes ao desenvolvimento da aplicação. Este levantamento foi realizado por meio de observação... cotidiano. 






\begin{itemize}
	\item escolha da dupla;
	\item Decisão do tema (dezembro 2014)
	\item Escolha do orientador
	\item Levantamento das tecnologias que seriam utilizadas
	\item Estudo e pequenos testes a fim de validar as tecnologias
	\item Definição da metodologia de desenvolvimento de software.
	\item Pesquisa de mercado para saber a aceitação da ideia
	\item Inicio do desenvolvimento do pré projeto
	\item Escrita da introdução, objetivos e justificativa
	\item levantamento de requisitos textuais do sistema ( através de possíveis cenários e simulações);
	\item Escrita do quadro teórico para o pré projeto 
	\item Quadro metodológico
	\item Criação do cronograma de atividades;
	\item Revisão do pré projeto
	\item Banca de qualificação
	\item Correção do projeto de acordo com as sugestões da banca
	\item atualização do cronograma	
	\item Iniciou-se o processo de elaboração da engenharia de software.
	\item Modelagem da base de dados
	\item Inicio do desenvolvimento do software.	
	\item Configuração do ambiente de desenvolvimento (Instalação das ferramentas)
	\item Tentativa de desenvolvimento utilizando a seguintes tecnologias:
		- Neo4j embedded, primefaces e JSF.(criar conta) Porem não estava fluindo como o esperado, uma vez que o Neoo4j utilizado da maneira embedded não nos permitia conectar ao sistema de gerenciamento da base de dados (web)e abrir uma nova instancia de conexão simultânea devido a limitações do próprio Neo4j, já que um processo java já ocupava tal conexão com o socket.
		Outro problema encontrado ao utilizar tais tecnologias foi que tanto a parte cliente (front end) quanto a parte do servidor (back end, modelo de negócios) se encontravam totalmente acoplados em uma aplicação java web, portanto, a cada alteração realizada nos controllers (principalmente, pois este era alterado constantemente, uma vez que ele era diretamente relacionado a página web,seguindo o padrão de desenvolvimento estabelecido ao utilizar as tecnologias descritas acima), sendo necessário recompilar, construir e publicar a aplicação no servidor web (tomcat). Isto estava impedindo que o desenvolvimento do software fluísse naturalmente, como era esperado. Por estes motivos decidiu-se mudar algumas das tecnologias utilizadas no front end e em como o banco de dados era acessado até então. 
			Passou a utilizar então:
			- Html5, CSS3, Java Script e o framework Angular JS para auxiliar no desenvolvimento do front end, ao invés de primefaces e JSF.
			- Para acesso ao banco de dados, lançou-se mão da forma embedded disponibilizada pelo banco de dados e passou-se a utilizar a API REST também disponibilizada pelo mesmo. Tais decisões nos permitiram desacoplar o sistema e manter o front end e o back end independentes.
		\item Estudo a respeito de web service REST;
		\item Instalação e configuração do Angular JS. 
		\item Testes utilizando o web Service REST a fim de validar a conexão com o banco de dados via API REST 
		\item Testes utilizando Angular JS para enviar requisições ao servidor e receber as respostas. 
		\item Configurar o acesso da aplicação cliente ao web service localizados em domínios distintos, utilizando o CORS. 
		\item Implementação do caso de uso 'criar conta' utilizando as novas tecnologias.
		\item Desenvolvido o sistema de login com armazenamento da sessão no cliente de forma errada; ( Colocando o usuário e senha decodificados na própria sessão do navegador).
		\item Desenvolvimento do Logoff
		\item Refatoração do sistema de login e armazenamento e validação da sessão. A partir deste ponto, foi utilizado o sistema de login via token. Gerando assim, sempre um novo token a fim de manter a segurança dos dados. 
		\item Criar a página inicial do sistema (Página Home)
		\item Implementação da busca por parceiros (buscando todos os contratantes registrados no banco, ou seja, sem critérios para a busca, a fim de criar a rede de parcerias primeiramente);
		\item Implementação da funcionalidade 'Adicionar Parceria'.
		\item Implementação da fila de requisição em espera para análise do contratante no menu principal.
		\item Implementação da funcionalidade 'Aceitar parceria'.
		\item Implementação do serviço de localização de possíveis parceiros com base em sua rede de relacionamento (parcerias) na página inicial do sistema.
		\item Implementação da busca por todos os parceiros do contratante autenticado no sistema para ser apresentado na página de rede de parceria
		\item Implementação da busca por novos parceiros para o contratante autenticado no sistema na página de rede de parceria.
		\item Implementação da funcionalidade 'Listar todos os serviços' cujo o provedor de serviço possui atrelado a ele.
		\item Implementação da funcionalidade 'Adicionar serviços' para os provedores de serviços.
		\item Implementação de remoção de serviços.
		\item Implementação da funcionalidade Listar Mão de obra. Inicialmente, como feito na lista de parceiros, também feito na lista de provedores de serviços. (Listei todos os provedores de serviços que executavam o serviço procurado pela pessoa naquele momento), a fim de criar a rede de avaliações de serviços.
		\item Implementar a funcionalidade de Avaliar o serviço.
		\item Implementar a funcionalidade de Localização de mão de obra, agora baseando-se nas avaliações anteriormente realizadas pelos meus parceiros.
		\item Abranger a busca por provedores de serviço, que ainda não foram avaliadas, para localizar mão de obras dentro da mesma cidade do contratante.
		\item Implementar a busca pelas últimas avaliações realizadas pelo contratante autenticado no sistema.
		\item Criar a página de perfil do provedor de serviço e apresentar um relatório contendo a média das avaliações em diferentes situações (na rede de parceria, na empresa e na cidade)
		\item Criar a página de perfil dos contratantes apresentando a quantidade de parceiros em comum e o nome de alguns deles
		\item Desenvolver o design do sistema
		\item Refatorar a página Incial (Página Home) de ambos os tipos de acesso (Provedor de serviços e Contratantes ou ambos)
		\item Criar uma página com dicas randômicas para os prestadores de serviços
		\item Criar o feed de notícias na página inicial do sistema (parecido com o do facebook)
		\item Criar os gráficos (à desenvolver)	
\end{itemize}

\par Realizando todos os passos descritos acima, será possível obter como resultado final a realização deste projeto.


%Comentar, pois na correção do pré-projeto a Joelma disse que isto seria feito via tópicos e não texto.
%\par Para desenvolver este projeto, usaremos os processos de engenharia de softwares definidos no ICONIX, que consistem em quatro etapas subdivididas em algumas tarefas específicas.

%\par Na primeira etapa, será realizada a análise dos requisitos que serão necessários para o início do projeto. Nesta fase será desenvolvido o modelo de domínio e, posteriormente, os casos de uso.

%\par Após o término da primeira etapa, será feita a análise preliminar, realizando para cada caso de uso, um diagrama de robustez e atualizando paralelamente o modelo de domínio, com os atributos e métodos. Com as informações geradas a partir deste processo, será possível desenvolver a base de dados da aplicação.

%\par Após a segunda etapa concluída será criado, para cada caso de uso, um diagrama de sequencia, contendo detalhes a respeito da futura implementação e novamente, o modelo de domínio gerado na 1ª etapa será atualizado, incluindo a ele os novos métodos que foram coletados neste processo.

%\par Para concluir, o último passo será a implementação, que com base em todos os processos descritos acima, será realizado com segurança, de que este projeto possivelmente irá atender aos requisitos levantados.
