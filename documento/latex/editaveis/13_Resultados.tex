
\chapter{DISCUSSÃO DOS RESULTADOS} 

\par Neste capitulo serão apresentados e discutidos os resultados obtidos por esta pesquisa e desenvolvimento do \textit{software}, apresentando os seus pontos positivos e negativos.

\par Inicialmente foram utilizadas as tecnologias Java, Primefaces e JSF para desenvolver este trabalho, no entanto, viu-se a necessidade de alterar algumas destas tecnologias, visando agilizar o processo de desenvolvimento. Desta forma, passou-se a utilizar HTML, CSS, Javascript em conjunto com o \textit{framework} Angular JS.

\par A mudança de tecnologias trouxe como benefício o desacoplamento das partes cliente (\textit{front-end}) e servidor (\textit{back end}), não sendo mais necessário recompilar, construir e publicar a aplicação no servidor \textit{web}, como era feito até então a cada alteração.

\par Após esta mudança, notou-se que a forma como o banco de dados era acessado poderia ser ajustada, a fim de seguir a mesma ideia proposta pela troca de tecnologias já mencionadas, uma vez que o banco de dados Neo4j permite duas maneiras de conexão, sendo elas: \textit{embedded} e por meio da API REST.(XXXXX).Desta forma, passou-se a utilizar a API REST ao invés da forma \textit{embedded}, utilizada até então.

\par Posterior a estas mudanças, deu-se continuidade no desenvolvimento da aplicação, onde os seus principais resultados da são apresentados a seguir.




\section{Neo4j}

\par Para desenvolver este trabalho foi necessário utilizar um banco de dados orientado a grafo, a fim de aplicar o mesmo conceito utilizado pelas redes sociais, para que o objetivo proposto por este trabalho fosse alcançado. Foi realizada um levantamento para validar as opções de banco de dados orientado a grafos disponíveis, entretanto, o Neo4j demonstrou-se superior aos demais pela sua simplicidade de instalação, utilização, configuração e por possuir uma linguagem para realizar consultas que se assemelha muito à linguagem humana, a API \textit{Cypher}.

\par No inicio do desenvolvimento deste projeto esperava-se que o Neo4j fosse utilizado por meio da forma \textit{embedded}, porém, devido a dificuldades que esta forma apresentou, uma vez que, não era possível conectar-se à base de dados pela aplicação em desenvolvimento e pelo sistema de gerenciamento do Neo4j simultaneamente, impedindo que as alterações no banco de dados fossem validadas pelo sistema de gerenciamentono momento em que as alterações eram realizadas no \textit{software} desenvolvido neste trabalho, somado ao fato que foi necessário alterar as tecnologias que inicialmente esperavam ser utilizadas no \textit{front-end}, houve-se a necessidade de lançar mão da forma \textit{server} e utilizar a API REST que o próprio Neo4j disponibiliza e abrir mão antiga maneira proposta inicialmente.

\par Após estas definições foram criados testes para validar a conexão ao banco de dados, a fim de constatar que a conexão com o Neo4j via API REST em conjunto com a também API \textit{Cypher} havia sido realizada de forma correta.

\par Na próxima seção serão apresentados os resultados obtidos no desenvolvimento da aplicação \textit{web}.

\section{Aplicação web}

\par A linguagem Java foi proposta desde o princípio para ser utlizada como linguagem padrão para o \textit{back-end}, porém devido aos problemas já citados relacionados a troca de tecnologias, foi proposto realizar a troca da linguagem Java por PHP, pois, apesar dos autores deste trabalho obterem conhecimento nesta linguagem, o PHP não foi abordado no curso o que seria muito bem aceito, entretanto, foi realizado alguns testes e constatou-se que, esta troca não seria conveniente devido a excassez de documentação que esta nova linguagem apresenta se tratanto da comunicação com o Neo4j via API REST. Sendo assim, optou-se por continuar a utilizar o Java, o que evitou o despendimento de tempo para resolver problemas com falta de documentação.

\par Ainda após as trocas de tecnologias e  da forma de conectar-se ao banco de dados, foi necessário alterar as classes responsáveis pela conexão com o Neo4j e também as classes que realizam as operações no banco. 

