
\chapter{DISCUSSÃO DOS RESULTADOS} 

\par O desenvolvimento deste trabalho teve como objetivo desenvolver um sistema que pudesse auxiliar em contratações de mão de obra temporária, fazendo uso do banco do dados orientado a grafos. Por meio do desenvolvimento do \textit{software}, utilizando as linguagens de programação HTML, CSS, Javascript e Java, juntamente com o \textit{framework} Angular JS, foi possível desenvolver este sistema, iniciando primeiramente com o objetivo de analisar os conceitos e as funcionalidades do banco de dados orientado a grafos.

\par Web service (comunicação com o banco de dados)
\par Telas de navegação (mudanças de tecnologia)
\par Banco de dados (comentar sobre a mudança na maneira de utilização)
\par Linguagem Java (falta de documentação em PHP);


\par Telas de navegação

\par Segundo Shackel (1991), a usabilidade de um sistema pode ser definida como a possibilidade do sistema ser utilizado facilmente e eficazmente pela gama de usuários aos quais ele se destina na realização de determinadas tarefas, dentro de contextos específicos.  O design de uma tela é parte integrante de um produto online e é de extrema importância no planejamento visual. Por meio dele é possível criar uma experiencia amigável ao usuário. Para auxiliar no desenvolvimento das telas, foram escolhidas tecnologias  


\section{Web Service}

\par Inicialmente, foi-se proposto desenvolver um sistema para a busca de mão de obra e para tanto, não se fazia necessário utilizar um \textit{web service}, uma vez que, o banco de dados que foi utilizado fornecia duas maneiras de se conectar a ele, sendo elas a \textit{embedded} e a segunda por meio da API REST fornecida por ele, conforme (XXXX). A princípio seria utilizado a primeira forma, o que facilitaria o desenvolvimento do \textit{software}, pois ele seria desenvolvido utilizando tecnologias como o Java, o JSF e Primefaces o que permitiria ao desenvolvedor trabalhar diretamente com objetos sem a necessidade de um meio de comunicação externo entre o \textit{front-end} e \textit{back-end}.

\section{Neo4j}

\par Para desenvolver este trabalho foi necessário utilizar um banco de dados orientado a grafo, a fim de aplicar o mesmo conceito utilizado pelas redes sociais, para que o objetivo proposto por este trabalho fosse alcançado. Foi realizada um levantamento para validar as opções de banco de dados orientado a grafos disponíveis, entretanto, o Neo4j demonstrou-se superior aos demais pela sua simplicidade de instalação, utilização, configuração e por possuir uma linguagem para realizar consultas que se assemelha muito à linguagem humana, a API \textit{Cypher}.

\par No inicio do desenvolvimento deste projeto esperava-se que o Neo4j fosse utilizado por meio da forma \textit{embedded}, porém, devido a dificuldades que esta forma apresentou, uma vez que, não era possível conectar-se à base de dados pela aplicação em desenvolvimento e pelo sistema de gerenciamento do Neo4j simultaneamente, impedindo que as alterações no banco de dados fossem validadas pelo sistema de gerenciamentono momento em que as alterações eram realizadas no \textit{software} desenvolvido neste trabalho, somado ao fato que foi necessário alterar as tecnologias que inicialmente esperavam ser utilizadas no \textit{front-end}, houve-se a necessidade de lançar mão da forma \textit{server} e utilizar a API REST que o próprio Neo4j disponibiliza e abrir mão antiga maneira proposta inicialmente.

\par Após estas definições foram criados testes para validar a conexão ao banco de dados, a fim de constatar que a conexão com o Neo4j via API REST em conjunto com a também API \textit{Cypher} havia sido realizada de forma correta.

\par Na próxima seção serão apresentados os resultados obtidos no desenvolvimento da aplicação \textit{web}.

\section{Aplicação web}

\par A linguagem Java foi proposta desde o princípio para ser utlizada como linguagem padrão para o \textit{back-end}, porém devido aos problemas já citados relacionados a troca de tecnologias, foi proposto realizar a troca da linguagem Java por PHP, pois, apesar dos autores deste trabalho obterem conhecimento nesta linguagem, o PHP não foi abordado no curso o que seria muito bem aceito, entretanto, foi realizado alguns testes e constatou-se que, esta troca não seria conveniente devido a excassez de documentação que esta nova linguagem apresenta se tratanto da comunicação com o Neo4j via API REST. Sendo assim, optou-se por continuar a utilizar o Java, o que evitou o despendimento de tempo para resolver problemas com falta de documentação.

\par Ainda após as trocas de tecnologias e  da forma de conectar-se ao banco de dados, foi necessário alterar as classes responsáveis pela conexão com o Neo4j e também as classes que realizam as operações no banco. 

