\subsection{Banco de dados NoSQL}

\par A expressão NoSQL é um termo não definido claramente. Ela foi ouvida pela primeira vez em 1998 como um nome para o banco de dados relacional de Carlo Strozzi, que assim o nomeou por não fornecer uma SQL-API. O mesmo termo foi usado como nome do evento NoSQL Meetup em 2009, que teve como objetivo a discussão sobre sistemas de bancos de dados distribuídos.

\par Devido à explosão de conteúdos na \textit{web} no início do século XXI, houve a necessidade de substituir os bancos de dados relacionais por bancos que oferecessem maior capacidade de otimização e performance, a fim de suportar o grande volume de informações eminentes a esta mudança \cite{bruggen_learning_neo4j}.

\par \citeonline[p. 27]{rocha_algoritmos_particionamento_banco_dados_orientado_grafos} afirma que NoSQL é "um acrônimo para Not only SQL, indicando que esses bancos não usam somente o recurso de Structured Query Language (SQL), mas outros recursos que auxiliam no armazenamento e na busca de dados em um banco  não relacional".

\par Segundo \citeonline{bruggen_learning_neo4j}, os banco de dados NoSQL podem ser categorizados de 4 maneiras diferentes, são elas: \textit{Key-Value stores}\footnotemark[7], \textit{Column-Family stores}\footnotemark[8], \textit{Document stores}\footnotemark[9] e \textit{Graph Databases}\footnotemark[10].

\footnotetext[7]{\textit{Key-Value stores} - armazenamento por um par de chave e valor.}

\footnotetext[8]{\textit{Column-Family stores} - armazenamento por colunas e linhas.}

\footnotetext[9]{\textit{Document stores} - armazenamento em arquivos.}

\footnotetext[10]{\textit{Graph Databases} - banco de dados orientado a grafo.}
%\begin{itemize}
%\item \textit{Key-Value stores};
%\item \textit{Column-Family stores};
%\item \textit{Document stores};
%\item \textit{Graph Databases}.
%\end{itemize}

\par De acordo com \citeonline{bruggen_learning_neo4j}, o banco de dados orientado a grafo (\textit{graph database}) pertence a categoria NoSQL, contudo, ele possui particularidades que o torna muito diferente dos demais tipos de bancos de dados NoSQL. A seguir, será descrito com maiores detalhes o banco de dados orientado a grafos Neo4j.
