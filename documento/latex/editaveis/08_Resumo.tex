% --- resumo em português ---

\begin{OnehalfSpacing} 

\noindent \imprimirAutorCitacaoMaiuscula. {\bfseries\imprimirtitulo}. {\imprimirdata}.  Monografia -- Curso de {\MakeUppercase\imprimircurso}, {\imprimirinstituicao}, {\imprimirlocal}, {\imprimirdata}.

\vspace{\onelineskip}
\vspace{\onelineskip}
\vspace{\onelineskip}
\vspace{\onelineskip}

\begin{resumo}
~\\
%início do texto do resumo
\noindent Esta pesquisa acadêmica apresenta o estudo do banco de dados orientado a grafos aplicado à busca por mão de obra temporária, utilizando o mesmo modelo de negócio presente nas redes sociais. Esta pesquisa é do tipo aplicada, pois visa propor uma solução a um problema social, que acontece dentro de uma região específica, mas que pode ser útil a toda população. Para obter os resultados, foi utilizado o banco de dados Neo4j, juntamente com a API \textit{Cypher}, usada para a navegação e inserção de dados no banco. Foi desenvolvido um sistema na plataforma \textit{web}, com o auxílio das tecnologias HTML, CSS Javascript e o \textit{framework} Angular JS, atuando do lado cliente da aplicação e as tecnologias Java, \textit{Web service} REST, juntamente com o Neo4j, atuando do lado servidor. Foram desenvolvidas funcionalidades que beneficiam tanto o prestador de serviços quanto o contratante, auxiliando-os na tomada de decisão. Como exemplo, pode-se citar a busca por determinado tipo de serviço, avaliação de desempenho de profissionais, criação de redes de parcerias, além de gráficos comparativos, que mostram ao usuário informações sobre as últimas avaliações. Essas funcionalidades tornaram esta pesquisa de grande relevância social, técnica e teórica por contribuir com a exploração de novos recursos, relacionados aos bancos de dados NoSQL. Com isso, conclui-se que os objetivos propostos por este trabalho foram atendidos.


%fim do texto do resumo
\vspace{\onelineskip}
\vspace*{\fill}
\noindent \textbf{Palavras-chave}: \imprimirPalavraChaveUm. \imprimirPalavraChaveDois. \imprimirPalavraChaveTres.
\vspace{\onelineskip}
\end{resumo}

\end{OnehalfSpacing}
