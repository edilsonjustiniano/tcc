\subsection{\textit{Cypher Query Language}}

\par O \textit{Cypher Query Language} é uma linguagem para consultas em banco de dados orientado a grafo específica para o banco Neo4j. Ela foi criada devido à necessidade de manipular os dados e realizar buscas em grafos de uma forma mais simples, uma vez que, não é necessário escrever \textit{traversals} (\textit{travessias}) para navegar pelo grafo \cite{neo4j_team_manual}.


\par \citeonline{robinson_webber_eifrem_graph_databases}, afirmam que, o \textit{Cypher} foi desenvolvido para ser uma \textit{query language} que utiliza uma linguagem formal, permitindo a um ser humano entendê-la. Desta forma, qualquer pessoa envolvida no projeto é capaz de compreender as consultas realizadas no banco de dados. 

%BANCA_QUALIFICACAO. Comentado este parágrafo, porém o mesmo retornará para a banca de qualificação
\par Segundo \citeonline{neo4j_team_manual}, o \textit{Cypher} foi inspirado em uma série de abordagens e construído sob algumas práticas já estabelecidas, inclusive a SQL. Por este motivo, é possível notar que ele utiliza algumas palavras reservadas que são comuns na SQL como \textit{WHERE} e \textit{ORDER BY}.

%BANCA_QUALIFICACAO. Comentado este parágrafo, porém o mesmo retornará para a banca de qualificação
\par De acordo com \citeonline{neo4j_team_manual}, o \textit{Cypher} é composto por algumas cláusulas, dentre elas, se destacam:

%%BANCA_QUALIFICACAO. Comentado estes itens, porém os mesmos retornarão para a banca de qualificação
\begin{itemize}
	\item \textit{START}: Define um nó inicial para a busca.
	\item \textit{MATCH}: Define o padrão de correspondência entre os nós.
	\item \textit{CREATE}: Cria nós e relacionanemtnos.
	\item \textit{WHERE}: Define um critério de busca.
	\item \textit{RETURN}: Define quais nós e/ou atributos devem ser retornados da \textit{query} realizada. 
\end{itemize} 

\par Outras \textit{Query Languages} existem, inclusive com suporte ao Neo4j, porém devido as vantagens apresentadas acima, somada ao fato que ele possui uma curva de aprendizagem menor e é excelente para lhe oferecer uma base a respeito de grafos,  este \textit{framework} será utilizado para realizar as tarefas de manipulação dos dados no banco de dados.
