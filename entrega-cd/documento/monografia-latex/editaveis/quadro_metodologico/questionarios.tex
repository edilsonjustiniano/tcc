\subsection{Questionários}

\par Para \citeonline{silva_menezes_metodologia_pesquisa_elaboracao_dissertacao}, questionário é:

\begin{citacao}
	Uma série ordenada de perguntas que devem ser respondidas por escrito pelo informante. O questionário deve ser objetivo, limitado em extensão e estar acompanhado de instruções As instruções devem esclarecer o propósito de sua aplicação, ressaltar a importância da colaboração do informante e facilitar o preenchimento.
\end{citacao}

%\par Qual o objetivo de aplicar o questionário? Serão aplicados os questionários para a empresa tal, a fim de levantar o grau de satisfação dos funcionários com o sistema.

%\par Quantos ou quantas e para quem?

\par Serão disponibilizados questionários \textit{on-line} a fim de avaliar o interesse da população por um sistema que ofereça o serviço de localização de mão de obra temporária. Para desenvolver tal \textit{software}, uma bateria de testes será feita visando oferecer a melhor interação entre o usuário e o software, levando em conta sua aplicabilidade, desempenho e facilidade de utilização.

%\par A observação será feita através de formulários de pesquisa que avaliarão o interesse dos usuários pelo sistema que ofereça o serviço de localização de mão de obra temporária. Os testes serão feitos visando oferecer a melhor interação entre o usuário e o software, levando em conta sua aplicabilidade, desempenho e facilidade de utilização.


%Reunião não é aplicado ao nosso trabalho pq reunião não é um instrumento de pesquisa
%\section{Reuniões}

%\par Serão realizados encontros presenciais e também virtuais com os acadêmicos e com o professor orientador, para o levantamento dos requisitos, da aplicabilidade do software, dos modelos de engenharia de software e codificação do sistema, como também questões teóricas que estejam relacionadas com as melhores práticas para se obter um software aplicável e ágil.
