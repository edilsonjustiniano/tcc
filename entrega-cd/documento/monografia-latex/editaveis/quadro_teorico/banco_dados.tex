\subsection{Banco de dados}

\par A expressão ''banco de dados'' teve origem a partir do termo inglês \textit{Databanks}, que foi substituído, mais tarde, pela palavra \textit{Databases} (Base de dados)  por possuir um significado mais apropriado \cite {setzer_silva_banco_dados_aprenda_o_que_sao_melhore_conhecimento}.

\par De acordo com \citeonline{date_introducao_sistemas_bancos_dados}, um banco de dados é uma coleção de dados persistentes, usada pelos sistemas de aplicação em uma determinada empresa. Sendo assim, um banco de dados é um local onde são armazenados os dados necessários para manter as atividades de determinadas organizações.

\par Um banco de dados possui, implicitamente, as seguintes propriedades: representa aspectos do mundo real; é uma coleção lógica de dados que possuem um sentido próprio e armazena dados para atender uma necessidade específica. O tamanho do banco de dados pode ser variável, desde que ele atenda às necessidades dos interessados em seu conteúdo \cite{elmasri_navathe_sistemas_banco_dados}.

\par A escolha do banco de dados que será utilizado em um projeto é uma decisão importante e que deve ser tomada na fase de planejamento, pois determina características da futura aplicação, como a integridade dos dados, o tratamento de acesso de usuários, a forma de realizar uma consulta e o desempenho. Portanto, essa decisão deve ser bem analisada, levando-se em consideração o tipo de \textit{software} e o ambiente de produção em que será utilizado.

\par Nas seções seguintes, são demonstrados os principais modelos de banco de dados, abordando suas características.