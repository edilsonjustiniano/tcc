\section{Instrumentos}

\par Os instrumentos de pesquisa são as ferramentas usadas para a coleta de dados. Como afirma \citeonline[p. 117]{marconi_lakatos_metodologia_trabalho_cientifico}, os instrumentos de pesquisa abrangem “desde os tópicos de entrevista, passando pelo questionamento e formulário, até os testes ou escala de medida de opiniões e atitudes”. Eles são de suma importância para o desenvolvimento de um projeto, pois visam levantar o máximo de informações possíveis para nortear as tomadas de decisões.

\par Para a realização deste projeto foram utilizados questionários, reuniões e análise documental como instrumentos de pesquisa. 

\par Segundo \citeonline[p. 31]{gil_como_elaborar_projeto_de_pesquisa}, o questionário é um dos procedimentos mais utilizados para obter informações, pois é uma técnica de custo razoável, apresenta as mesmas questões para todas as pessoas, garante o anonimato e pode conter as questões para atender a finalidades especificas de uma pesquisa. Se aplicada criteriosamente, esta técnica apresenta elevada confiabilidade. Podem ser desenvolvidos para medir opiniões, comportamento, entre outras questões, também pode ser aplicada individualmente ou em grupos.

\par Para este trabalho, foi desenvolvido um questionário informal, aplicado de forma individual, disponibilizado por meio da internet. Esse questionário foi respondido por pessoas de diferentes perfis sociais, com o intuito de analisar se o projeto no qual essa pesquisa visa desenvolver seria bem aceito. A seguir são discriminados os resultados obtidos.

Aqui colocar o resultado da análise do questionário

\par A análise documental consiste em identificar, verificar e apreciar os documentos com uma finalidade específica, visando extrair uma informação objetiva da fonte. Para este trabalho foram utilizadas a análise documental como auxiliar teórico e também como consulta em dúvidas do desenvolvimento prático.


